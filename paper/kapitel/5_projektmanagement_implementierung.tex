\documentclass[../main.tex]{subfiles}

\begin{document}

\subsection{Entwicklungsmethodik}

\subsubsection{Agile Arbeitsweise}

Die Agile Arbeitsweise repräsentiert eine revolutionäre Verschiebung in der Methodik der Softwareentwicklung, die zu Beginn des 21.
Jahrhunderts entstand. Ihre Ursprünge lassen sich auf das `Agile Manifesto' zurückführen, das von einer Gruppe von Softwareentwicklern
verfasst wurde. Das Manifest legt den Fokus auf Flexibilität, Kundenorientierung und die kontinuierliche Auslieferung von Software. Es
stellt einen Gegenentwurf zu den traditionellen, starren und plangetriebenen Entwicklungsmodellen dar, die sich oft als ineffizient in
der schnelllebigen und veränderlichen Welt der Softwareentwicklung erwiesen haben.

Die Agile Arbeitsweise umfasst verschiedene Prinzipien und Praktiken, die die enge Zusammenarbeit im Team, die Anpassungsfähigkeit an
Veränderungen, die kontinuierliche Verbesserung und die häufige Auslieferung von funktionierenden Softwareteilen betonen. Zu den
Kernprinzipien gehören die Priorisierung von Kundenbedürfnissen, die Arbeit in iterativen und inkrementellen Zyklen (Sprints),
Selbstorganisation und Cross-Funktionalität der Teams sowie die regelmäßige Reflexion zur Optimierung der Arbeitsprozesse.

Auf der Grundlage der agilen Prinzipien haben sich verschiedene Arbeitsweisen wie Scrum und Kanban entwickelt, die spezifische
Rahmenwerke und Methoden bieten, um Agile in der Praxis umzusetzen. Scrum fokussiert sich auf die Organisation der Teamarbeit in
Sprints mit festen Rollen (wie Product Owner und Scrum Master), Artefakten (wie Product Backlog und Sprint Backlog) und regelmäßigen
Meetings (Daily Scrum, Sprint Review, Sprint Retrospective). Kanban hingegen betont die Visualisierung der Arbeit durch ein Kanban-Board,
die Limitierung laufender Arbeiten und die kontinuierliche Lieferung, ohne feste Iterationen.

Die Vorteile der Agilen Arbeitsweise sind vielfältig. Sie ermöglicht eine höhere Reaktionsfähigkeit auf Kundenwünsche und Marktanforderungen,
führt zu einer verbesserten Produktqualität durch kontinuierliches Feedback und ermöglicht eine effizientere Ressourcennutzung durch adaptive
Planung. Darüber hinaus fördert sie eine stärkere Einbindung und Motivation der Teammitglieder, indem sie Eigenverantwortung und direkte
Kommunikation in den Vordergrund stellt.

Für das Projekt `ShroomScout' wurde sich aufgrund bereits vorhandener Arbeitserfahrungen für eine agile Herangehensweise entschieden. Die
Entscheidung basiert auf der Erkenntnis, dass agile Methoden eine flexible und effektive Rahmenstruktur bieten, um den Herausforderungen
und Dynamiken der Softwareentwicklung gerecht zu werden. Die Agilität ermöglicht es dem Team, schnell auf Änderungen zu reagieren, kontinuierlich
Wert zu liefern und dabei den Entwicklungsprozess stetig zu optimieren. Diese Erfahrungen und die Überzeugung von den Vorteilen agiler Prinzipien
führten zur Implementierung agiler Praktiken als grundlegende Entwicklungsstrategie für `ShroomScout', um ein hochwertiges, benutzerzentriertes
Produkt in einem kollaborativen und adaptiven Entwicklungsprozess zu realisieren.

\subsubsection{Software Kanban Board}

Software Kanban ist eine Methode des Lean Managements, die ursprünglich in der Produktion bei Toyota entwickelt wurde und mittlerweile weitreichende
Anwendung in der Softwareentwicklung findet. Das Kernstück dieser Methode ist das Kanban Board, ein visuelles Werkzeug zur Darstellung von Arbeitsabläufen.
Ein Kanban Board ist typischerweise in mehrere Spalten unterteilt, die verschiedene Phasen des Arbeitsprozesses repräsentieren, von `Zu tun' über `In Arbeit'
bis hin zu `Erledigt'. Jede Aufgabe, repräsentiert durch eine Karte (oder `Issue'), wird durch die Spalten bewegt, um ihren Fortschritt durch den Arbeitsprozess
zu visualisieren. Diese Methode unterstützt Teams dabei, den Arbeitsfluss zu optimieren, Engpässe zu identifizieren und die Effizienz zu steigern.

Im Rahmen des `ShroomScout'-Projekts wurde ein Kanban Board auf GitHub als Project implementiert. GitHub bietet die Möglichkeit, solche Boards direkt in den
Projektkontext zu integrieren. Beide Teammitglieder hatten Zugriff auf das Board, was eine flexible und transparente Zusammenarbeit ermöglichte. Die Nutzung
eines Kanban Boards auf GitHub erlaubte es dem Team, selbstständig Issues zu erstellen, diese entsprechenden Personen zuzuweisen und ihren Status zu aktualisieren,
sobald Aufgaben begonnen oder abgeschlossen wurden.

Ein wesentlicher Vorteil dieser Herangehensweise ist die Möglichkeit, in Git Commit Messages direkt auf Issues zu verweisen. Dies ermöglichte eine nahtlose
Nachverfolgbarkeit von Codeänderungen zu spezifischen Aufgaben und förderte ein tiefgreifendes Verständnis darüber, welche Änderungen aus welchem Grund
vorgenommen wurden. Diese Praxis erhöht die Codequalität und erleichtert die Fehlersuche und -behebung.

Die Anwendung eines Software Kanban Boards via GitHub brachte zahlreiche Vorteile für das `ShroomScout'-Projekt mit sich. Zu diesen Vorteilen zählten:

\begin{itemize}

	\item \textbf{Transparenz:}
	      Jedes Teammitglied hatte jederzeit einen vollständigen Überblick über den aktuellen Projektstatus, wer an welchen Aufgaben arbeitet und welche
	      Issues noch bearbeitet werden müssen. Diese Transparenz förderte das gemeinsame Verständnis und die Zusammenarbeit im Team, sowie das individuelle
	      Zeitmanagement.

	\item \textbf{Übersichtliche Planung:}
	      Die visuelle Darstellung des Arbeitsprozesses erleichterte die Priorisierung von Aufgaben und die Planung kommender Arbeitsschritte.
	      Engpässe und ungenutzte Ressourcen konnten schnell identifiziert und adressiert werden.

	\item \textbf{Flexibilität:}
	      Die Möglichkeit, Issues flexibel zwischen den Spalten zu verschieben, unterstützte das Team dabei, auf Veränderungen im Projektumfeld oder neue
	      Ideen für Anforderungen reagieren zu können, ohne den Überblick zu verlieren.

	\item \textbf{Verbesserte Kommunikation:}
	      Durch die Zuweisung von Aufgaben und die direkte Referenzierung von Issues in Commit Messages wurde die Kommunikation innerhalb
	      des Teams sowie die Dokumentation des Entwicklungsprozesses verbessert.

\end{itemize}

Insgesamt trug die Anwendung von Software Kanban und die Nutzung eines Kanban Boards auf GitHub maßgeblich zur Effizienz, Transparenz und Flexibilität des
Entwicklungsprozesses bei `ShroomScout' bei und stellte eine solide Basis für eine erfolgreiche Projektumsetzung dar.

\subsubsection{Anwendung von Scrum Artefakten \& Events}

Im Rahmen des `ShroomScout'-Projekts wurde eine angepasste Form der Scrum-Methodik angewendet, um die agile Entwicklung zu unterstützen. Scrum, eine der
prominentesten agilen Methoden, ist bekannt für seine strukturierte Herangehensweise an die Softwareentwicklung, die iterative Arbeit in Sprints, regelmäßige
Abstimmungsmeetings und eine klare Rollenverteilung innerhalb des Teams vorsieht. Eine zentrale Komponente von Scrum sind die täglichen Stand-up-Meetings, auch
als Daily Scrums bekannt, die dazu dienen, den Fortschritt zu überprüfen und den weiteren Arbeitsablauf effektiv zu koordinieren.

Für das Projekt wurde diese Praxis durch die Einführung eines täglichen 15-minütigen Anrufs, des Daily Calls, adaptiert. Diese kurzen, fokussierten Meetings
dienten als Plattform, um Aktualisierungen zu teilen, den Fortschritt zu diskutieren und Herausforderungen sowie die Planung für den kommenden Tag zu besprechen.
Trotz der Kürze waren diese Calls ausreichend, um eine kontinuierliche Abstimmung zwischen den Teammitgliedern zu gewährleisten und sicherzustellen, dass alle
auf dem gleichen Stand sind.

Die Kombination des Daily Calls mit dem Einsatz eines Kanban Boards auf GitHub erwies sich als besonders effektiv. Während das Kanban Board eine kontinuierliche
visuelle Übersicht über den Status aller Aufgaben bot, ermöglichten die Daily Calls eine dynamische Anpassung an Veränderungen und die sofortige Klärung von
Fragen. Diese Synergie förderte nicht nur die Transparenz und Flexibilität im Entwicklungsprozess, sondern trug auch dazu bei, die Prinzipien der agilen Entwicklung,
wie schnelle Anpassungsfähigkeit an Veränderungen und verbesserte Kommunikation, effektiv umzusetzen.

Die Anwendung dieser angepassten Scrum-Methodik im `ShroomScout'-Projekt illustriert, wie agile Praktiken an die spezifischen Bedürfnisse und Rahmenbedingungen
eines Projekts angepasst werden können. Die täglichen Calls stärkten die Teamdynamik und förderten eine Kultur der Offenheit und des kontinuierlichen Lernens.
In Kombination mit dem Kanban Board ermöglichte diese Vorgehensweise eine agile, reaktionsfähige und effiziente Projektarbeit, die maßgeblich zum erfolgreichen
Abschluss von `ShroomScout' beitrug.

\subsection{Qualitaetssicherung \& Testing}

\subsubsection{Anwendung von Clean Code Prinzipien}

Clean Code bezeichnet eine Sammlung von Prinzipien und Praktiken für die Softwareentwicklung, die darauf abzielen, den Quellcode lesbar, verständlich und wartbar
zu gestalten. Außerdem führt die Einhaltung von Clean Code Prinzipien zu einer höheren Codequalität und erleichtert die Zusammenarbeit in Teams. Vorteile umfassen
in der Theorie unter anderem die Reduzierung von Fehlern, eine effizientere Einarbeitung neuer Teammitglieder und eine gesteigerte Produktivität bei der Weiterentwicklung
der Software. Folgende grundlegende Prinzipien des Clean Code wurden im Projekt berücksichtigt:

\begin{itemize}

	\item \textbf{Klarheit und Verständlichkeit:}
	      Der Code sollte so geschrieben sein, dass er auch von anderen Entwicklern leicht verstanden werden kann. Dies beinhaltet die Verwendung aussagekräftiger
	      Namen für Variablen, Funktionen und Klassen sowie eine klare Strukturierung des Codes. Ein Beispiel dafür findet sich etwa in der Eingabemaskenkomponente
	      (`register.component.ts'):

	      \begin{verbatim}
          // Schlecht
          protected t: string = 'Pilzdaten angeben:';
          protected env: string[] = ['Wiese', 'Eiche', 'Buche'];
          protected sEnv: string = '';

          // Gut
          protected title: string = 'Pilzdaten angeben:';
          protected environments: string[] = ['Wiese', 'Eiche', 'Buche'];
          protected selectedEnvironment: string = '';
        \end{verbatim}

	\item \textbf{Kurze Funktionen und Klassen:}
	      Funktionen und Klassen sollten eine einzige Aufgabe erfüllen und möglichst kurz gehalten werden. Dies trägt zur Lesbarkeit und Wiederverwendbarkeit bei.
	      Ein Beispiel stellt die Kartenkomponente (`map.component.ts') dar:

	      \begin{verbatim}
          ngOnInit() {
            this.initializeMap();
            this.handleMapClicks();
          }
        \end{verbatim}

	      Der Inhalt der beiden Methoden hätte auch innerhalb der `ngOnInit()' Methode, welche von Angular beim initialisieren der Komponente aufgerufen wird,
	      eingefügt werden können. Dies hätte jedoch zu einer sehr langen und unüberischtlichen `ngOnInit()' Methode geführt.

	\item \textbf{Vermeidung von Code-Duplikationen:}
	      Jedes Codefragment sollte eine eindeutige, unmissverständliche und einmalige Funktion innerhalb des Systems haben. Das wird zum Beispiel bei der Einbindung
	      der Kartenkomponente deutlich. Da diese immer zu sehen ist, wird sie auch nur einmal in der `app.component.ts' eingebunden. Lediglich der Teil daneben ändert
	      sich in Bezug auf die URL.

	\item \textbf{Verwendung von Kommentaren:}
	      Kommentare sollten genutzt werden, um die Intention hinter dem Code zu erklären, nicht um zu beschreiben, was der Code tut. Der Code selbst sollte durch
	      seine Klarheit überzeugen. Dies wird ebenfalls in der Kartenkomponente deutlich. Da in der Methode die Karte bei bestimmten Koordinaten initialisiert wird,
	      stellt der Kommentar klar, dass es sich hierbei um Berlin handelt, nicht etwa wie das Laden genau funktioniert.

	      \begin{verbatim}
          /**
          * Initializes the leaflet map to Berlin's coordinates
          * and loads the actual map tiles from OSM.
          */
          private initializeMap(): void {...}
        \end{verbatim}

\end{itemize}

Die Anwendung von Clean Code Prinzipien im `ShroomScout'-Projekt trug wesentlich zur Qualitätssicherung und zur Effizienz der Softwareentwicklung bei. Durch
die konsequente Umsetzung dieser Prinzipien konnte ein Code erstellt werden, der nicht nur funktional zuverlässig ist, sondern auch eine hohe Lesbarkeit und
Wartbarkeit aufweist.

\subsubsection{Unit Tests}

In der Softwareentwicklung ist eine strukturierte Testhierarchie essenziell, um die Qualität und Zuverlässigkeit von Softwareprodukten zu gewährleisten. Diese
Hierarchie beginnt typischerweise mit Unit Tests, gefolgt von Integrationstests, Systemtests bis hin zu Akzeptanztests. Jede Testebene adressiert unterschiedliche
Aspekte der Software und hilft dabei, Fehler und Probleme frühzeitig im Entwicklungsprozess zu identifizieren.

Unit Tests stellen die Basis dieser Hierarchie dar und fokussieren sich auf die Überprüfung der kleinsten testbaren Teile einer Anwendung, in der Regel einzelne
Funktionen oder Methoden. Durch die Isolation spezifischer Codeabschnitte ermöglichen Unit Tests die Überprüfung der Funktionalität und Korrektheit einzelner
Komponenten unabhängig vom restlichen System.

Die Vorteile von Unit Tests sind vielfältig:

\begin{itemize}

	\item \textbf{Frühe Fehlererkennung:}
	      Unit Tests helfen, Fehler und Probleme bereits in der Entwicklungsphase zu erkennen, bevor sie tiefgreifende Auswirkungen auf das Gesamtsystem haben.

	\item \textbf{Dokumentation:}
	      Sie dienen als lebendige Dokumentation des Codes, indem sie dessen erwartetes Verhalten beschreiben.

	\item \textbf{Designverbesserung:}
	      Die Notwendigkeit, Codeeinheiten isoliert zu testen, fördert saubere und lose gekoppelte Codearchitekturen.

\end{itemize}

Für das Projekt `ShroomScout' wurde sich ausschließlich für die Implementierung von Unit Tests entschieden. Diese Entscheidung basiert auf der Erkenntnis,
dass umfangreichere Testformen wie Integrationstests oder Performance Tests ohne eine Produktionsumgebung und angesichts der ständigen Weiterentwicklung des
Projekts nur begrenzten Mehrwert bieten würden. Unit Tests hingegen er\-mög\-lich\-en eine effektive und effiziente Überprüfung der Kernfunktionalitäten von
`ShroomScout' und unterstützen eine agile Entwicklungsmethodik.

\textit{Codebeispiele zur Verdeutlichung der Zwecke von Unit Tests:}

\begin{verbatim}

...

\end{verbatim}

Die Anwendung von Unit Tests in `ShroomScout' spielte eine zentrale Rolle in der Qua\-li\-täts\-sich\-e\-rung, indem sie die Zuverlässigkeit einzelner
Funktionen und Methoden kontinuierlich überprüfte. Dies trug wesentlich zur Stabilität und Wartbarkeit des Projekts bei und stellte sicher, dass auch bei
zukünftigen Erweiterungen und Anpassungen die Integrität der Anwendung gewahrt bleibt.

\end{document}
