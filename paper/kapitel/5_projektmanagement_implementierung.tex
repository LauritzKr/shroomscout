\documentclass[../main.tex]{subfiles}
%~2200 Worte
\begin{document}
\subsection{Entwicklungsmethodik} %Lauritz
\subsubsection{Agile Arbeitsweise}
Die Agile Arbeitsweise repräsentiert eine revolutionäre Verschiebung in der Methodik der Softwareentwicklung, die zu Beginn des 21. 
Jahrhunderts entstand. Ihre Ursprünge lassen sich auf das "Agile Manifesto" zurückführen, das im Jahr 2001 von einer Gruppe von 
Softwareentwicklern verfasst wurde. Das Manifest legt den Fokus auf Flexibilität, Kundenorientierung und die kontinuierliche Auslieferung 
von Software. Es stellt einen Gegenentwurf zu den traditionellen, starren und plangetriebenen Entwicklungsmodellen dar, die sich oft als 
ineffizient in der schnelllebigen und veränderlichen Welt der Softwareentwicklung erwiesen haben.

Die Agile Arbeitsweise umfasst verschiedene Prinzipien und Praktiken, die die enge Zusammenarbeit im Team, die Anpassungsfähigkeit an 
Veränderungen, die kontinuierliche Verbesserung und die häufige Auslieferung von funktionierenden Softwareteilen betonen. Zu den 
Kernprinzipien gehören die Priorisierung von Kundenbedürfnissen, die Arbeit in iterativen und inkrementellen Zyklen (Sprints), 
Selbstorganisation und Cross-Funktionalität der Teams sowie die regelmäßige Reflexion zur Optimierung der Arbeitsprozesse.

Auf der Grundlage der agilen Prinzipien haben sich verschiedene Arbeitsweisen wie Scrum und Kanban entwickelt, die spezifische 
Rahmenwerke und Methoden bieten, um Agile in der Praxis umzusetzen. Scrum fokussiert sich auf die Organisation der Teamarbeit in 
Sprints mit festen Rollen (wie Product Owner und Scrum Master), Artefakten (wie Product Backlog und Sprint Backlog) und regelmäßigen 
Meetings (Daily Scrum, Sprint Review, Sprint Retrospective). Kanban hingegen betont die Visualisierung der Arbeit durch ein Kanban-Board, 
die Limitierung laufender Arbeiten und die kontinuierliche Lieferung, ohne feste Iterationen.

Die Vorteile der Agilen Arbeitsweise sind vielfältig. Sie ermöglicht eine höhere Reaktionsfähigkeit auf Kundenwünsche und Marktanforderungen, 
führt zu einer verbesserten Produktqualität durch kontinuierliches Feedback und ermöglicht eine effizientere Ressourcennutzung durch adaptive 
Planung. Darüber hinaus fördert sie eine stärkere Einbindung und Motivation der Teammitglieder, indem sie Eigenverantwortung und direkte 
Kommunikation in den Vordergrund stellt.

Für das Projekt "ShroomScout" wurde sich aufgrund bereits vorhandener Arbeitserfahrungen für eine agile Herangehensweise entschieden. Die 
Entscheidung basiert auf der Erkenntnis, dass agile Methoden eine flexible und effektive Rahmenstruktur bieten, um den Herausforderungen 
und Dynamiken der Softwareentwicklung gerecht zu werden. Die Agilität ermöglicht es dem Team, schnell auf Änderungen zu reagieren, kontinuierlich 
Wert zu liefern und dabei den Entwicklungsprozess stetig zu optimieren. Diese Erfahrungen und die Überzeugung von den Vorteilen agiler Prinzipien 
führten zur Implementierung agiler Praktiken als grundlegende Entwicklungsstrategie für "ShroomScout", um ein hochwertiges, benutzerzentriertes 
Produkt in einem kollaborativen und adaptiven Entwicklungsprozess zu realisieren.

\subsubsection{Software Kanban Board}
\subsubsection{Anwendung von Scrum Artefakten \& Events}

\subsection{Herausforderungen \& Bewaeltigungsstrategien} %Tim
\subsubsection{Open Streetmap Datenbank}

\subsection{Qualitaetssicherung \& Testing} %Lauritz
\subsubsection{Anwendung von Clean Code Prinzipien}
\subsubsection{Unit Tests}
\end{document}
