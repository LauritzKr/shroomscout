\documentclass[../main.tex]{subfiles}
%~2200 Worte
\begin{document}
\subsection{Systemarchitektur} %Tim
\subsubsection{Architekturmuster}
\subsubsection{Entscheidungsfindung \& spezifische Auswahl}

\subsection{Datenbankdesign} %Tim
\subsubsection{Datenmodell}
\subsubsection{Datenbanksystemauswahl}

\subsection{Backend-Design} %Tim
\subsubsection{Notwendige API-Abfragen}
\subsubsection{Backend-Framework Entscheidung}

\subsection{GUI-Design \& Frontend-Frameworks} %Lauritz
\subsubsection{Gestaltungsprozess}
Der Designprozess der grafischen Benutzeroberfläche (GUI) begann mit der Erstellung einfacher Diagramme auf 
draw.io (siehe Bild). Diese initiale Skizzierung fokussierte sich zunächst auf die Startseite, welche die Karte und den Live 
Feed umfasste, und wurde später um die Eingabeseite für das Eintragen eines Pilzes erweitert. Diese konzeptionelle Phase 
erfolgte in einer gemeinsamen Brainstorming-Session, in der Ideen frei ausgetauscht und Entwürfe entwickelt und verglichen 
wurden.

[ BILD DER DRAW.IO DIAGRAMME ]

Nicht-funktionale Anforderungen wie Einfachheit, Benutzerfreundlichkeit und Responsive Design waren hierbei maßgebliche 
Faktoren, an denen sich der Designprozess orientierte. Die Einfachheit der Benutzeroberfläche wurde als essenziell erachtet, 
um die Nutzerführung intuitiv und zugänglich zu gestalten. So wurde jedes Element der GUI, sowie deren Anordnung, mit dem Ziel 
entworfen, eine direkte und unkomplizierte Interaktion möglich zu machen. Benutzerfreundlichkeit stand ebenfalls im Vordergrund, 
um sicherzustellen, dass Nutzer aller Erfahrungsstufen die Anwendung problemlos verwenden können. So gibt es keine überflüssigen
Elemente und die vorhandenen sind gut sichtbar und selbsterklärend. Die Anpassung an verschiedene Endgeräte durch ein Responsive 
Design war ein weiterer kritischer Aspekt, der ebenfalls Berücksichtigung fand. Da die Applikation höhst wahrscheinlich schon 
während des Pilzesuchens verwendet wird, gewährleistet das Responsive Design, dass "ShroomScout" auf Desktops, Tablets und 
Smartphones gleichermaßen funktionell und ästhetisch ansprechend ist.

\subsubsection{Frontend-Framework Entscheidung}
Nach der Gestaltung der GUI stand letztlich noch die Auswahl eines Frontend-Frameworks an. Ein Frontend-Framework ist eine Sammlung 
wiederverwendbarer Designvorlagen und Code-Snippets, die Entwicklern helfen, konsistente und effiziente Benutzeroberflächen 
zu erstellen. Diese Frameworks bieten vorgefertigte Komponenten und Werkzeuge, die die Entwicklung beschleunigen und die 
Einhaltung von Webstandards erleichtern.

Die Entscheidung fiel recht schnell auf Angular, hauptsächlich aufgrund von bereits vorhandenen Erfahrungen und aufgebautem
Basiswissen mit der Technologie. Obwohl andere Frontend-Frameworks wie React und Vue ebenfalls in Betracht gezogen wurden, 
gab die Vertrautheit mit Angular den Ausschlag. React, bekannt für seine Flexibilität und Leistung, und Vue, geschätzt für 
seine Einfachheit und leichtes Lernen, sind ebenfalls hervorragende Optionen für die Entwicklung moderner Webanwendungen. Die 
Entscheidung für Angular basierte jedoch auf der spezifischen Projekterfahrung und dem Komfortniveau der Entwickler. 

Darüber hinaus überzeugt Angular durch eine komponentenbasierte Architektur und dadurch modulare Herangehensweise bei der Entwicklung.
Jeder Baustein der Anwendung wird dabei in einer Komponente gekapselt, was Klarheit der Anwendungsstruktur fördert und die Wiederverwendung
dieser Komponenten möglich macht. Signifikante Vorteile umfassen weiterhin die Bereitstellung umfangreicher Bibliotheken von eingebauten 
Funktionen und Diensten, wie Formularverwaltung, Routing oder ein HTTP-Client, sowie eine verbesserte Entwicklungserfahrung durch die 
Verwendung von TypeScript. TypeScript erweitert JavaScript um starke Typisierung und objektorientierte Programmierkonzepte wie Klassen, 
Interfaces und Dekoratoren. Letztlich ermöglicht Angular durch seine umfangreiche Dokumentation und Community-Unterstützung eine effiziente 
Problemlösung und Weiterentwicklung.

\end{document}
