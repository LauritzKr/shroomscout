\documentclass[../main.tex]{subfiles}

\begin{document}

\subsection{Motivation \& Zielbeschreibung}

Die Entwicklung von `ShroomScout', einer Webanwendung auf Angular-Basis, entsprang einer tiefen Leidenschaft
für das Pilzesammeln. Diese Freizeitbeschäftigung verbindet die Freude am Erkunden der Natur mit dem praktischen
Nutzen der Ernte essbarer Pilze aus ihrem natürlichen Umfeld. Die Erkenntnis, dass Pilze unter bestimmten Bedingungen
gedeihen und recht zuverlässig an denselben Stellen wiederkehren, motivierte zur Schaffung einer digitalen Plattform,
welche die Kartierung und gemeinschaftliche Nutzung von Pilzfundorten ermöglichen soll.

`ShroomScout' bedient sich einer interaktiven, auf OpenStreetMap basierten Karte, um Nutzern eine visuelle
Grundlage zum Eintragen der Standorte ihrer Pilzfunde bieten zu können. Die Benutzeroberfläche ist intuitiv gestaltet:
Mit einem einfachen Klick auf die Schaltfläche `Pilz eintragen' öffnet sich ein Eingabefenster, das nicht nur die
Eingabe des Pilznamens mittels automatischer Vervollständigung ermöglicht, sondern auch die Beschreibung der Umgebung,
in der der Pilz gefunden wurde. Ein visuelles Element in Form eines Bildes des Pilzes soll den Nutzer dabei unterstützen,
die korrekte Identifikation sicherzustellen, was besonders bei der Unterscheidung von essbaren und potenziell giftigen
Pilzen von unschätzbarem Wert ist.

Die Kernfunktionalität von `ShroomScout' wird durch einen Live-Feed ergänzt, der es den Nutzern ermöglicht,
in Echtzeit zu sehen, welche Pilze kürzlich und wo gefunden wurden. Diese Funktion fördert nicht nur den
Austausch innerhalb der Pilzsammlergemeinschaft, sondern trägt auch dazu bei, das Bewusstsein und das
Interesse am Pilzesammeln zu steigern --- eine Praxis, die in der heutigen Zeit vielleicht nicht mehr so
verbreitet ist wie in der Vergangenheit.

Das übergeordnete Ziel von `ShroomScout' ist also das Schaffen einer Plattform, die Pilzsammler bei der Dokumentation
ihrer Funde unterstützt, die Sicherheit beim Sammeln erhöht und letztlich eine Gemeinschaft Gleichgesinnter
aufbaut. Dabei bietet die Applikation eine wertvolle Ressource, die es ermöglicht, basierend auf der kollektiven Erfahrung
und dem Wissen der Nutzer fundierte Entscheidungen über die Identität und Sicherheit von Pilzen zu treffen. Durch die
Bereitstellung einer solchen Plattform soll das Interesse und Wissen um das Pilzesammeln gefördert und somit einen Beitrag
zur Wiederbelebung dieser traditionellen Aktivität geleistet werden.

\subsection{Projektmethodik \& Herangehensweise}

Die Planungsphase des Projekts `ShroomScout' begann mit der Ideenfindung und der Identifikation essenzieller Use Cases,
die das Interagieren mit der Karte, das Eintragen eines Pilzes und die Verfolgung des Live-Feeds umfassten. Diese initialen
Überlegungen führten zur detaillierten Skizzierung aller Aspekte der grafischen Benutzeroberfläche (GUI), um einen klaren
Aufbau und eine intuitive Nutzerführung sicherzustellen. Die Planung mündete in einem Architekturentwurf, der eine Dreiteilung
in Frontend, Backend und eine Datenbank für die Speicherung der Koordinaten der eingetragenen Pilze vorsah. Nach Festlegung
dieser Struktur erfolgte die Auswahl der Technologien: Angular für das Frontend, um an vorhandene Kenntnisse anzuknüpfen,
FastAPI mit Python für das Backend, basierend auf positiven Erfahrungen in einem vorangegangenen Universitätsprojekt, und
SQLite als Datenbanklösung aufgrund ihrer Einfachheit und Effizienz.

Die Umsetzung des Projekts erfolgte nach Vorbild von agilen Prinzipien, wie Scrum oder Software Kanban. Die Nutzung von GitHub
ermöglichte eine effiziente Aufgabenverwaltung auf einem Kanban-Board, wodurch eine kontinuierliche Entwicklung und Priorisierung
von Features gewährleistet wurden. Tägliche Abstimmungen im Rahmen von Daily Calls, angelehnt an die Scrum-Methodik, förderten die
Kommunikation und Problemlösung im Team. Der Einsatz von Git unterstützte die verteilte Entwicklung und sicherte einen reibungslosen
Integrationsprozess der einzelnen Komponenten. Die Arbeitsteilung erfolgte klar definiert in zum einen Frontend und zum anderen
Backend/Datenbank, um an bisheriges Wissen und individuelle Präferenzen anzuknüpfen, sowie klare, getrennte Verantwortlichkeiten zu
schaffen.

Diese methodische Herangehensweise und Arbeitsteilung ermöglichten es, `ShroomScout' systematisch von der Konzeption bis zur
funktionsfähigen Anwendung zu entwickeln. Die agile Vorgehensweise mit regelmäßigen Abstimmungen und einer flexiblen Anpassung
an aufkommende Anforderungen und Herausforderungen bildete das Fundament für den erfolgreichen Projektverlauf. Die Entscheidung
für vertraute Technologien trug zusätzlich zur Effizienz des Entwicklungsprozesses bei und ermöglichte eine fokussierte Umsetzung
der geplanten Funktionalitäten.

\subsection{Voraussichtliche Risiken}

\subsubsection{Projektbezogene Risiken}

Die Entwicklung einer Webanwendung ist ein komplexes Unterfangen, das verschiedene projektbezogene Risiken birgt. Diese Risiken
können, wenn sie nicht angegangen werden, die rechtzeitige Fertigstellung des Projekts beeinträchtigen. Im Folgenden werden die
vier wesentlichsten Risiken und deren voraussichtlich eingeplanten Bewältigungsstrategien dargestellt.

\begin{itemize}

	\item \textbf{Kommunikation und Zusammenarbeit im Team:}
	      Eines der Hauptprobleme in der Entwicklung ist die fehlende Kommunikation und Zusammenarbeit im Team. Diese kann zu Merge
	      Konflikten und inkompatiblem Code führen, dessen Behebung zusätzliche Zeit in Anspruch nimmt. Um dieses Risiko zu minimieren,
	      wurden tägliche Abstimmungsgespräche (Daily Calls) eingeführt, in denen der Fortschritt diskutiert und Probleme frühzeitig
	      identifiziert werden konnten. Dies förderte die Transparenz und Zusammenarbeit im Team und half, vorhandene Konflikte schnell
	      zu lösen und voraussichtliche zu vermeiden.

	\item \textbf{Mangel an spezifischem Fachwissen:}
	      Ein weiteres Risiko stellt der Mangel an spezifischem Fachwissen dar, der zu einer zu langen Einarbeitungszeit führen kann. Eine
	      so einfache wie effektive Bewältigungsstrategie war hier die Arbeitsteilung nach Erfahrung und Kenntnissen der Teammitglieder. Indem
	      Aufgaben entsprechend den individuellen Fähigkeiten zugewiesen wurden, konnte die Produktivität gesteigert und die Einarbeitungszeit
	      reduziert werden.

	\item \textbf{Verfügbarkeit von Teammitgliedern:}
	      Die Verfügbarkeit von Teammitgliedern ist ebenfalls ein kritisches Risiko, welches das Projekt verzögern kann. Um diesem Risiko zu begegnen,
	      war eine flexible und frühzeitige Planung essentiell. Durch eine klare Arbeitsteilung seit Beginn des Projekts wurde das persönliche
	      Zeitmanagement massiv erleichtert, da für jedes Teammitglied klar war, welcher Teil des Projekts zu erledigen ist.

	\item \textbf{Datenverlust:}
	      Schließlich ist Datenverlust ein ernsthaftes Risiko, sei es durch technische Probleme wie einem Ausfall von GitHub oder den Verlust eines
	      Laptops. Um Datenverlust zu verhindern, war es wichtig, regelmäßig Code-Änderungen in Git zu sichern und die Repositories auf GitHub stets
	      auf dem aktuellen Stand zu halten. Dies stellte Unabhängigkeit vom lokalen Code auf den Laptops sicher, da der neuste Stand immer zentralisiert
	      in Git zur Verfügung stand.

\end{itemize}

\subsubsection{Technologische Risiken}

Neben projektbezogenen Risiken erwachsen aus dem hochkomplexen Feld der modernen Software Entwicklung auch Technologische Risiken. Einige dieser Risiken
sind im Folgenden aufgeführt.

\begin{itemize}

	\item \textbf{Sicherheitslücken:}
	      Webanwendungen bieten aufgrund diverser Eigenschaften, nicht zuletzt der Dependency Struktur im Javascript Ökosystem, eine breite Angriffsfläche
	      für böswillige Akteure. Unter anderem deshalb wurde sich gegen ein Hosting der Website entschieden und die Entwicklung ausschließlich lokal durchgeführt.

	\item \textbf{Git:}
	      Git ist ein etablierter Standard in der Software Entwicklung, kann jedoch auch zu Problemen führen. Ein klassisches Beispiel sind falsch gelöste Merge
	      Konflikte beim zusammenführen divergenter Versionen der Software. Aufgrund der Funktionsweise führt dies in den seltensten Fällen zu schweren Problemen,
	      kann die Entwicklung jedoch signifikant verlangsamen.

	\item \textbf{Externe Infrastruktur:}
	      In der Entwicklung moderner Software begibt man sich meist zwangsläufig in Abhängigkeit von externer Infrastruktur. Im Falle von Shroomscout bestehen,
	      unter anderem, starke Abhängigkeiten zur Verfügbarkeit von Github und der NPM-Paket Quellen. Sollte diese Infrastruktur scheitern könnte es die weitere
	      Entwicklung der Software erschweren, verzögern oder gar unmöglich machen.

\end{itemize}

\end{document}
