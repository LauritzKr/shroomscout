\section{Fazit}

\subsection{Anforderungskonformität der Applikation}

Das wichtigste Kriterium für gute Software ist Anforderungskonformität. Im folgenden werden die Anforderungen den Ergebnissen gegenübergestellt:

\subsubsection{Funktionale Anforderungen}

\begin{itemize}

	\item \textbf{Anforderung: Registrieren eines Pilzfunds}
	      Der Nutzer muss die Möglichkeit haben einen neuen Pilzfund über eine Eingabemaske im Frontend zu registrieren. Diese Eingabemaske soll eine
	      automatische Vervollständigung der Eingabe anbieten und soll bereits validieren, ob die Eingabe einem gültigen Wert entspricht. Die in die
	      Eingabemaske einzugebenden Daten sollen die Art des Pilzes, sowie die Umgebung, in der er gefunden wurde, sein. Erst wenn ein Marker für den
	      Standort auf der Karte gesetzt und sowohl für Art und Umgebung des Pilzes valide Werte eingetragen wurden, soll es möglich sein den Eintrag zu
	      speichern. Davor soll der Button zum registrieren ausgegraut und somit nicht klickbar sein.

	\item \textbf{Umsetzung: Registrieren eines Pilzfunds}
	      Die Eingabemaske wurde entsprechend der Anforderungen umgesetzt. Die Validierung der vom Nutzer eingegeben Werte wurde mit Hilfe von Angular
	      FormControl und der Angular Materials Komponente `Autocomplete' umgesetzt.

	\item \textbf{Anforderung: Anzeigen eines Live Feeds}
	      Es soll auf der Startseite einen Livefeed geben, in dem die letzten Pilzfunde aufgeführt sind. Aus dem Livefeed sollen Art und Umgebung der
	      zuletzt registrierten Pilze hervorgehen.

	\item \textbf{Umsetzung: Anzeigen eines Live Feeds}
	      Der Live Feed wurde mit Hilfe eines Angular Services und einer Angular Materials Komponente umgesetzt.

	\item \textbf{Anforderung: Anzeige der registrierten Pilze}
	      Die registrierten Pilze sollen auf einer OpenStreetMap Karte als Marker angezeigt werden. Wenn über die Marker gehovert wird, sollen zusätzliche
	      Informationen, genauer, der Name des Pilzes, angezeigt werden.

	\item \textbf{Umsetzung: Anzeige der registrierten Pilze}
	      Die registrierten Pilze werden anhand ihrer Längen- und Breitengraddaten in der Datenbank gespeichert. Beim Laden der Seite werden die Informationen
	      Name, Umgebung, Längengrad und Breitengrad von einem Angular Service beim Backend abgefragt, anschließend an das Frontend gesendet und als Marker auf
	      der Karte angezeigt.

	\item \textbf{Anforderung: Navigationsleiste}
	      Die Navigationsleiste soll die Navigation innerhalb der Webseite ermöglichen und die Möglichkeit bieten, die Eingabemaske zum Registrieren eines
	      Pilzes zu öffnen, sowie zur Startseite zurückzukehren. Die Navigationsleiste soll blau sein und am linken Bildschirmrand das ShroomScout Logo
	      beinhalten.

	\item \textbf{Umsetzung: Navigationsleiste}
	      Die Navigationsleiste wurde mit Hilfe der Angular Material `Navbar' Komponente implementiert und ermöglicht den Zugriff auf alle Teile des Frontends.
	      Die Navigationsleiste ist blau und beinhaltet das Shroomscout Logo am linken Bildschirmrand.

	\item \textbf{Anforderung: Design}
	      Um eine einheitliche Erfahrung zu schaffen und die Wartung zu erleichtern, sollen primär Komponenten der von Google verwalteten Angular Materials
	      Bibliothek verwendet werden.

	\item \textbf{Umsetzung: Design}
	      Bis auf die Karte, wurde das Design der Website ausschließlich mit Angular Material Komponenten erstellt.

\end{itemize}

\subsubsection{Nicht-funktionale Anforderungen}

\begin{itemize}

	\item \textbf{Anforderung: Einfachheit und Benutzerfreundlichkeit}
	      `ShroomScout' soll durch eine intuitive Bedienbarkeit bestechen. Die Anwendung muss so gestaltet sein, dass Nutzer mit
	      minimalen Aufwand und ohne Vorkenntnisse die Kernfunktionalitäten nutzen können. Dazu gehört eine klare und verständliche
	      Nutzeroberfläche, die es dem Anwender ermöglicht, ohne umständliche Navigation oder komplizierte Prozesse Pilze zu finden
	      und einzutragen.

	\item \textbf{Umsetzung: Einfachheit und Benutzerfreundlichkeit}
	      Die Einfachheit wurde durch ein klares, minimales Design umgesetzt. Um die Benutzung für alle Nutzer zu erleichtern, wurden
	      Buttons mit großer deskriptiver Aufschrift eingesetzt.

	\item \textbf{Anforderung: Responsive Design}
	      Angesichts der Tatsache, dass Nutzer `Shroom\-Scout' häufig im Freien und damit auf mobilen Geräten nutzen werden, ist ein
	      responsive Design unerlässlich. Das heißt, dass sich die Anwendung automatisch an verschiedene Bildschirmgrößen und -auflösungen
	      anpassen muss, um auf Smartphones, Tablets und Desktop-Computern gleichermaßen gut bedienbar zu sein. Dies gewährleistet eine
	      optimale Benutzererfahrung unabhängig vom Endgerät.

	\item \textbf{Umsetzung: Responsive Design}
	      Die Applikation wurde mithilfe von CSS auf eine verbesserte Anzeige auf mobilen Geräten optimiert.

	\item \textbf{Anforderung: Schnelle Ladezeiten}
	      Für eine positive Nutzererfahrung sind kurze Ladezeiten von großer Bedeutung. `ShroomScout' sollte so optimiert sein, dass die
	      Anwendung auch bei langsamer Internetverbindung schnell lädt, bzw. aktualisiert. Dies ist besonders wichtig, da Nutzer die Anwendung
	      möglicherweise in Gebieten mit schlechter Netzabdeckung verwenden.

	\item \textbf{Umsetzung: Schnelle Ladezeiten}
	      Um die Ladezeiten zu optimieren, wurde auf eine möglichst schlichte Seite gesetzt, welche mit Ausnahme der OpenStreeMap Karte
	      ohne externe Inhalte auskommt, welche erst geladen werden müssten.

	\item \textbf{Anforderungen: Sicherheit und Datenschutz}
	      Die Sicherheit persönlicher Daten und die Wahrung der Privatsphäre der Nutzer sind essenzielle Anforderungen. `ShroomScout'
	      muss sicherstellen, dass alle Nutzerdaten, insbesondere Standortinformationen und persönliche Informationen, gemäß den
	      geltenden Datenschutzrichtlinien behandelt und geschützt werden.

	\item \textbf{Umsetzung: Sicherheit und Datenschutz}
	      Die Sicherheit wurde so gut wie möglich durch die Implementierung von Maßnahmen zur Verhinderung von SQL-Injections, Verwendung aktueller
	      Software Versionen, sowie möglichst weniger Ab\-häng\-ig\-keit\-en umgesetzt. Der Datenschutz wurde in sofern umgesetzt, als dass Shroomscout keine
	      personenbezogenen Daten erhebt oder speichert und die Nutzung ohne Account möglich ist. Beim produkiven Einsatz der Software ist für datenschutz-
	      und sicherheitsgerechte Konfiguration der Server Sorge zu tragen.

	\item \textbf{Anforderung: Skalierbarkeit}
	      Die Anwendung muss in der Lage sein, mit einer zunehmenden Anzahl von Nutzern und Datenmengen zu skalieren. Dies stellt
	      sicher, dass `ShroomScout' auch bei wachsender Beliebtheit und steigenden Anforderungen stabil und performant bleibt.

	\item \textbf{Umsetzung: Skalierbarkeit}
	      Die Skalierbarkeit der Anwendung wurde durch die Trennung von Frontend und Backend umgesetzt wodurch sich beide Teile individuell skalieren lassen.
	      Jedoch ist die Skalierbarkeit aufgrund des aktuell verwendeten Datenbanksystems auf vertikale Skalierung begrenzt, insofern kein weiter Aufwand betrieben wird,
	      die Datenbank über mehrere Instanzen zu synchronisieren.

\end{itemize}

Abschließend lässt sich festhalten, dass sowohl die funktionalen, als auch die nicht funktionalen Anforderungen in ausreichendem Maße umgesetzt wurden.

\subsection{Ausblick auf zukuenftige Entwicklungen}

Die Entwicklung einer Software ist potentiell nie endgültig abgeschlossen. Es existieren also stets Optimierungspotentiale und potentielle weitere Features.
Würde die Entwicklung von Shroomscout unmittelbar weitergeführt werden, wären die folgenden Punkte die unmittelbaren Verbesserungspotenziale

\begin{itemize}

	\item \textbf{Performanteres Datenbanksystem:}
	      Die Verwendung eines performanteren serverbasierten Datenbanksystem bildet das Fundament für weitere Optimierungen, da die aktuelle Lösung nicht für
	      nutzungsintensive Produktivanwendungen ausgelegt ist und daher bei der Skalierung der Software schnell zum Flaschenhals werden würde. Der Migrationsaufwand
	      auf ein System, wie MySQL oder PostgreSQL, wäre des weiterem mit verhältnismäßig geringem Aufwand verbunden, da solange die Tabellenstruktur übernommen wird,
	      die nötigen Anpassungen am Code verhältnismäßig klein sind.

	\item \textbf{Containerrisierung \& Microservices:}
	      Die Software folgt aktuell dem Server-Client-Muster und ist darauf ausgelegt unmittelbar auf einem System zu Laufen. Zur Erleichterung des Deployments und
	      Erhöhung der Skalierbarkeit ist es sinnvoll, Container Virtualisierungslösungen, wie beispielsweise Docker oder Podman zu nutzen. In diesem Rahmen wäre ebenfalls
	      eine Anpassung der Systemarchitektur hin zu Microservices sinnvoll. Bei diesem Ansatz wird die Applikation in viele Microservices aufgespalten, die jeweils eine
	      oder wenige Aufgaben erfüllen und unabhängig von einander laufen. Die Vorteile dieses Architekturansatz liegt ebenfalls in der erhöhten Skalierbarkeit, jedoch
	      bietet er auch erleichterte Fehlerfindung durch getrennt laufende Services, die individuell getestet werden können.

	\item \textbf{Implementierung von Validierungen:}
	      Eine wichtige Optimierung im Hinblick auf die Sicherheit der Applikation ist die Implementierungen von Validierungen für Nutzereingaben und Anfragen an das Backend.
	      Die aktuellen Validierungen verhindern nur eine versehentliche Eingabe ungültiger Werte durch den Nutzer und SQL-Injection Attacken. Anfragen an das Backend hingegen
	      werden im Moment nicht validiert, was potentiell zu Sicherheitslücken führen kann. Die Implementierung dieser Validierungen ist mit verhältnismäßig geringem Aufwand
	      verbunden, es besteht jedoch das Risiko, dass nicht alle Fälle abgedeckt werden. Des Weiteren ist es im produktiven Einsatz wichtig, Techniken wie Ratelimiting zu
	      implementieren, um DDOS Attacken vorzubeugen.

	\item \textbf{Nutzersystem:}
	      Im Moment funktioniert Shroomscout weitestgehend anonym; ein Nutzer muss sich also nicht authentifizieren, um die Software zu nutzen. Dies hat einige Vorteile,
	      nicht zuletzt die vereinfachte Umsetzung von Datenschutzrichtlinien bei minimaler Erhebung von Nutzerdaten. Jedoch bringt die Implementierung eines Nutzersystems
	      viele Vorteile mit sich. Nutzer könnten Präferenzen oder Filter speichern und eine zwingende Registrierung zur Nutzung der Anwendung kann Helfen, Missbrauch der
	      Plattform zu verhindern.

	\item \textbf{Reimplementierung des Backendcodes:}
	      Das Backend wurde in Python implementiert, maßgebliche Kriterien für diese Entscheidung waren die Einfachheit und das breite Angebot von Ressourcen für die
	      Programmiersprache. Python ist jedoch, verglichen mit anderen Programmiersprachen, nicht sehr performant und ressourceneffizient. Daher kann es sinnvoll sein, das
	      Backend in einer performanteren Programmiersprache, wie Rust oder C\# zu reimplementieren. Da dies jedoch mit sehr hohem Aufwand verbunden ist, wäre eine derartige
	      Optimierung nur bei sehr hoher Nutzeranzahl und großen Mengen an Traffic sinnvoll.

\end{itemize}
