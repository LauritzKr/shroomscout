\documentclass[../main.tex]{subfiles}
%1000 Worte, Tim
\begin{document}
\subsection{Anforderungskonformität der Applikation}
\subsection{Ausblick auf zukuenftige Entwicklungen}
Die Entwicklung einer Software ist nie endgültig abgeschlossen, es existieren stets Optimierungspotentiale und potentielle weitere Features.
Würde die Entwicklung von Shroomscout unmittelbar weitergeführt werden, wären die folgenden Punkte die unmittelbaren Verbesserungspotenziale

\begin{itemize}

    \item \textbf{Performanteres Datenbanksystem}
        Die Verwendung eines performanteren serverbasierten Datenbanksystem bildet das Fundament für weitere Optimierungen, da die aktuelle Lösung nicht für userintensive
        Produktivanwendungen ausgelegt ist und daher bei der Skalierung der Software schnell zum Flaschenhals werden würde. Der Migrationsaufwand auf ein System wie MySQL oder PostgreSQL
        wäre, des Weiterem mit verhältnismäßig geringem Aufwand verbunden, da solange die Tabellenstruktur übernommen wird, die nötigen Anpassungen am Code verhältnismäßig klein sind. 

	\item \textbf{Containerrisierung \& Microservices}
	    Die Software folgt aktuell dem Server-Client-Muster und ist darauf ausgelegt unmittelbar auf einem System zu Laufen.
        Zur Erleichterung des Deployments und Erhöhung der Skalierbarkeit, ist es sinnvoll Container Virtualisierungslösungen, wie beispielsweise Docker oder Podman zu nutzen.
        In diesem Rahmen wäre ebenfalls eine Anpassung der Systemarchitektur hin zu Microservices sinnvoll, bei diesem Ansatz wird die Applikation in viele Microservices aufgespalten,
        die jeweils eine oder wenige Aufgaben erfüllen und unabhängig von einander laufen. Die Vorteile dieses Architekturansatz liegt ebenfalls in der erhöhten Skalierbarkeit, jedoch
        bietet er auch erleichterte Fehlerfindung durch getrennt laufende Services, die individuell getestet werden können.
        
  \item \textbf{Implementierung von Validierungen}
        Eine wichtige Optimierung im Hinblick auf die Sicherheit der Applikation ist die Implementierungen von Validierungen für User Input und Anfragen an das Backend.
        Die aktuellen Validierungen, verhindern nur eine versehentliche eingabe ungültiger Werte durch den Nutzer und SQLinjection Attacken. Anfragen das Backend hingegen werden
        im Moment nicht validiert, was potentiell zu Sicherheitslücken führen kann. Die Implementierung dieser Validierungen ist mit verhältnismäßig geringem Aufwand verbunden,
        es besteht jedoch das Risiko, dass nicht alle Fälle abgedeckt werden.

    \item \textbf{Usersystem}
        

  \item \textbf{Reimplementierung des Backendcodes}
        Der Live-Feed besteht aus einer Angular Material Card (`<mat-card>'). Angular Material ist eine Bibliothek an vorgefertigten GUI-Komponenten,
        welche bereits ansprechend aussehen. Diese wurden überall, wo es möglich war, verwendet, um sich nicht mehr um das CSS-Styling kümmern zu müssen.
        In der Komponente werden Nachrichten dargestellt, welche sie vom Message Service erhält. Das tut sie, indem sie mithilfe der async-Pipe (`| async') 
        das Nachrichten-Observable abonniert und somit über Änderungen automatisch informiert wird. 

  \item \textbf{Eingabemaske:}
        Die Eingabemaske besteht ebenfalls aus Angular Material Komponenten für die entsprechenden Eingabefelder und das Drop-Down-Menü. Außerdem wird
        hier beim Eintragen eines Pilzes zum Einen der Marker Service zum persistieren des Markers und der Message Service zum Absenden einer neuen 
        Nachricht an den Live-Feed genutzt.

\end{itemize}


\end{document}
