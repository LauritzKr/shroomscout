\documentclass[../main.tex]{subfiles}

\begin{document}

\subsection{Übergreifende Technologien}

\subsubsection{VSCode}

Visual Studio Code, kurz VSCode, ist eine integrierte Entwicklungsumgebung (IDE), die von Microsoft entwickelt wurde. Eine IDE ist eine
Art Texteditor, der umfassende Funktionen zur Softwareentwicklung unter einer Oberfläche bündelt. Die Entscheidung für VSCode als primäre
Entwicklungsumgebung für `ShroomScout' fiel aufgrund mehrerer Faktoren. Zum einen bringen beide Entwickler bereits umfassende Erfahrung
mit VSCode aus früheren Projekten mit, was eine effiziente Arbeit ermöglicht. Zum anderen bietet VSCode eine einfache Erweiterbarkeit durch
zahlreiche Extensions. Diese Erweiterungen erleichtern die Entwicklung erheblich, indem sie zusätzliche Funktionen wie Syntaxhervorhebung,
Codevervollständigung und direkte Integration mit Git bereitstellen.

\subsubsection{Git \& GitHub}

Git ist ein weit verbreitetes System zur Versionskontrolle, das es Entwicklern ermöglicht, Änderungen am Quellcode nachzuverfolgen
und die Zusammenarbeit in Entwicklungsteams zu erleichtern. Git unterstützt die verteilte Arbeit, indem es jedem Entwickler ermöglicht,
eine vollständige Kopie des Repositorys lokal zu führen, inklusive der gesamten Historie der Codeänderungen. Für `ShroomScout' wurde
Git genutzt, um den Entwicklungsprozess zu verwalten und Änderungen effizient zu koordinieren. Das Projekt wurde auf GitHub gehostet,
einer Plattform, die kostenlose Hosting-Services für Git-Repositorys bietet. GitHub erleichtert die Zusammenarbeit, indem es Werkzeuge
für die Problemverfolgung, Code-Reviews und die Verwaltung von Pull Requests bereitstellt. Die Nutzung von GitHub als zentrale Plattform
trug zur reibungslosen Koordination zwischen den Entwicklern bei und ermöglichte eine transparente und nachvollziehbare Projektentwicklung.

\subsubsection{Open Street Map}

Open Street Map (OSM) ist ein kollaboratives Projekt zur Erstellung einer frei bearbeitbaren Weltkarte. Im Gegensatz zu kommerziellen
Kartenanbietern wie Google Maps bietet OSM den Vorteil, dass die Daten unter einer offenen Lizenz stehen, was die freie Nutzung und
Anpassung der Karteninformationen ermöglicht. Ein entscheidender Vorteil von OSM ist dabei die Möglichkeit, eigene Layer zu erstellen und
zu verwalten, etwa für Pilzfundorte. Diese Flexibilität und die offene Datenlizenz machen OSM zu einer idealen Wahl für Projekte, die auf
detaillierte geographische Informationen angewiesen sind und gleichzeitig die Kontrolle über die Darstellung und Verwaltung der Kartendaten
behalten möchten. Jedoch wurde sich im Projektverlauf gegen die Erstellung einer eigenen Pilzlayer und somit gegen das befüllen der OSM Datenbank
entschieden, um etwaige persönliche Konflikte mit den Betreibern dieser zu vermeiden. Des weiteren bietet eine eigene Datenbank mehr Flexibilität
und die Möglichkeit, das Backend ausgiebig mithilfe von Dummy-Testdaten zu testen. Dies wäre mit der offiziellen OSM Datenbank nicht möglich gewesen.

\subsection{Datenbankloesung}

Eine Datenbanklösung spielt für eine Webanwendung eine zentrale Rolle in der Verwaltung, Speicherung und Abfrage von Daten, die für die Funktion und
den Betrieb der Anwendung entscheidend sind. Sie dient als Rückgrat für die Datenspeicherung, indem sie einen strukturierten und effizienten Mechanismus
zur Datenhaltung bietet. Dies ermöglicht es, große Mengen von Daten systematisch zu organisieren, zu verwalten und zugänglich zu machen.

\subsubsection{SQLite}

SQLite ist ein weit verbreitetes, in C geschriebenes Datenbanksystem, das ein kompaktes, selbstgenügsames, hochzuverlässiges und vollständiges SQL-Datenbanksystem
bietet. Als eingebettete Datenbanksoftware ermöglicht SQLite die Speicherung von Daten in einer lokalen Datei, wodurch die Notwendigkeit eines separaten Servers
entfällt. Diese Eigenschaft macht SQLite optimal für kollaborative Software Entwicklung.

\subsubsection{Tabellenstruktur}

Die Datenbank für Shroomscout besteht aus 2 Tabellen, `findLog' und `mushrooms'. Die Tabelle `findLog' enthält alle Pilzfunde. Sie besitzt die Spalten: id, mushroom,
latitude, longitude und Environment; die id wird automatisch vergeben. Die restlichen Daten werden durch Nutzer, welche die Eingabemaske ausfüllen und einen Marker
durch Klicken auf die Karte setzen, eingegeben. Die Tabelle `mushrooms' enthält die Spalten id, name und scientificName. Die id wird erneut automatisch vergeben, Name
und wissenschaftlicher Name des Pilz werden manuell in die Datenbank eingepflegt.

\subsection{Backend Technologien}

\subsubsection{Python}

Python ist eine interpretierte, high-level Programmiersprache, die sich vor Allem durch ihre einfache Syntax und Übersichtlichkeit auszeichnet. Sie unterstützt
verschiedenste Programmierparadigmen und verfügt über eine umfangreiche Standardbibliothek sowie ein reiches Ökosystem von Drittanbieter-Bibliotheken. Dies macht
Python besonders attraktiv für die Entwicklung kleiner Webanwendungen wie `Shroomscout'.

\subsubsection*{FastAPI}

FastAPI ist ein hochperformantes, modernes Web-Framework für Python, das sich durch die Verwendung von Typ-Hinweisen auszeichnet und speziell für die Entwicklung
von APIs entworfen wurde. Es kombiniert Geschwindigkeit mit der Einfachheit der Validierung durch Python Type Hinting, was den Entwicklungsprozess beschleunigt und
die Fehleranfälligkeit reduziert. Durch die native Unterstützung asynchroner Anfragen verbessert FastAPI zudem die Effizienz und Skalierbarkeit von I/O-intensiven
Anwendungen. Ein weiterer Vorteil ist die automatische Erstellung von API-Dokumentationen mittels Swagger UI und ReDoc, welche die Nutzung und Wartung der API erheblich
erleichtert. Insgesamt bietet FastAPI eine leistungsstarke Kombination aus Geschwindigkeit, einfacher Handhabung und effizienter Entwicklung, was es zu einer guten Wahl
für Shroomscout macht.

\subsubsection{Uvicorn ASGI Server}

Uvicorn ist ein leichtgewichtiger, hochperformanter Application Server Gateway Interface (ASGI) Server, implementiert in Python. ASGI ist ein Spezifikationsstandard,
der als Schnittstelle zwischen asynchronen Python Web-Apps und Servern dient, ähnlich wie das Web Server Gateway Interface (WSGI) für synchrone Python Web-Apps.
Uvicorn nutzt die asynchrone und ereignisgesteuerte Programmierung, die in Python 3.5+ durch die Einführung von asyncio standardisiert wurde, um eine effiziente und
skalierbare Grundlage für das Hosting von Webanwendungen und -diensten zu bieten.

\subsection{Frontend Technologien}

\subsubsection{Angular \& Typescript}

Angular ist ein modernes, leistungsfähiges Frontend-Framework, das von Google entwickelt wurde und für die Erstellung dynamischer Single-Page
Applications (SPAs) eingesetzt wird. SPAs laden die notwendigen HTML-, JavaScript- und CSS-Codes bei der ersten Anfrage an den Server und
aktualisieren anschließend den Inhalt dynamisch ohne Seitenneuladung. Dieser Ansatz ermöglicht eine nahtlose Benutzererfahrung, ähnlich wie bei
einer Desktop-Anwendung, und bietet Vorteile wie verbesserte Performance, schnelle Reaktionszeiten und reduzierte Serveranfragen.

Des weiteren basiert Angular auf TypeScript, einer typisierten Superset-Sprache von JavaScript, die in JavaScript umgewandelt wird. Sie führt
objektorientierte Programmierkonzepte, wie optionale statische Typisierung ein, was zu einer besseren Codequalität, einfacherer Wartbarkeit und
erhöhter Entwicklerproduktivität führt. TypeScript erweitert JavaScript außerdem um zusätzliche Syntax für die Definition von Typen und unterstützt
somit die Entwicklung großer Anwendungen, indem es frühzeitige Fehlererkennung, bessere Refactoring-Fähigkeiten und eine autokomplette Funktion bietet.

\subsubsection{Komponenten \& Services}

Die Architektur von Angular ist komponentenbasiert, was bedeutet, dass Anwendungen aus Komponenten aufgebaut sind, die
jeweils einen Teil der Benutzeroberfläche darstellen. Diese Komponenten können unabhängig voneinander entwickelt, getestet und in der
Anwendung wiederverwendet werden, was die Entwicklungseffizienz und -geschwindigkeit erhöht. Eine Komponente in Angular besteht aus einer
Klasse, die mit einem @Component-Dekorator versehen ist. Dieser Dekorator definiert zugehörige Metadaten wie das Template (HTML), Styles (CSS)
und weitere Selektoren, wodurch eine Komponente die Kontrolle über einen bestimmten Bereich des Bildschirms in der Anwendung erhält.

Im Gegensatz zu Komponenten fokussieren sich Services in Angular auf die Implementierung von Geschäftslogik und die Bereitstellung von Daten.
Sie sind unabhängige Klassen, markiert mit einem @Injectable-Dekorator, und dienen dazu, wiederverwendbare Logik oder Datenzugriffsmethoden zu
kapseln. Services können in Komponenten oder anderen Services injiziert werden, wodurch eine lose Kopplung und eine höhere Wiederverwendbarkeit
der Codebestandteile gefördert wird.

Folgende Bestandteile der GUI wurden in Komponenten gekapselt:

\begin{itemize}

	\item \textbf{Navigationsleiste:}
	      Die Komponente `nav-bar' umfasst die Navigationsleiste am oberen Bildschirmrand. Sie ist immer zu sehen und bietet dem User die Möglichkeit,
	      über eine Schaltfläche zur Eingabemaske oder über das Logo zurück zur Startseite zu navigieren. Dazu wird lediglich zur URL `.../register' für die
	      Eingabemaske und zurück zur Root-URL für den Live-Feed navigiert. Im Routing Modul (`app-routing.module.ts') sind für diese Routen dann die
	      zugehörigen Komponenten deklariert, welche in der App Komponente (`app.component.ts') an Stelle des router-outlet-Tags angezeigt werden.
	      Dabei wird auch klar, dass die Karte immer zu sehen ist und lediglich der Teil daneben durch die entsprechende Komponente je nach der Route
	      befüllt wird.

	\item \textbf{Karte:}
	      In der Komponente `map' wird die Karte wird mithilfe des `leaflet'-Pakets (siehe `package.json') mit mit sogenannten `tiles', also kleinen
	      Teilen der Weltkarte, befüllt. Diese werden von der OpenStreetMap Tile-Provider URL geladen. Außerdem sorgt die Komponente dafür, dass die Karte
	      anklickbar wird, sollte die URL auf `/register' enden, sprich wenn die Eingabemaske angezeigt wird. Klick-Events durch den User werden vom Marker
	      Service verwaltet.

	\item \textbf{Live-Feed:}
	      Der Live-Feed besteht aus einer Angular Material Card (`<mat-card>'). Angular Material ist eine Bibliothek an vorgefertigten GUI-Komponenten,
	      welche bereits ansprechend aussehen. Diese wurden überall, wo es möglich war, verwendet, um sich nicht mehr um das CSS-Styling kümmern zu müssen.
	      In der Komponente werden Nachrichten dargestellt, welche sie vom Message Service erhält. Das tut sie, indem sie mithilfe der async-Pipe (`| async')
	      das Nachrichten-Observable abonniert und somit über Änderungen automatisch informiert wird.

	\item \textbf{Eingabemaske:}
	      Die Eingabemaske besteht ebenfalls aus Angular Material Komponenten für die entsprechenden Eingabefelder und das Drop-Down-Menü. Außerdem wird
	      hier beim Eintragen eines Pilzes zum Einen der Marker Service zum persistieren des Markers und der Message Service zum Absenden einer neuen
	      Nachricht an den Live-Feed genutzt.

\end{itemize}

Zwischen diesen Komponenten müssen an mehreren Stellen Daten ausgetauscht werden. Dazu dienen folgende Services:

\begin{itemize}

	\item \textbf{Message Service:}
	      Der Message Service ist für die im Live-Feed angezeigten Nachrichten zuständig. Somit enthält er eine Liste aller Nachrichten, sowie eine Methode,
	      welche zum Hinzufügen einer neuen Nachricht dient. Die aktuellen Nachrichten werden dann über einen sogennanten `Observable Stream', also einem
	      sich automatisch aktualisierenden Datenfluss bereitgestellt. Die Eingabemaske sendet eine neue Nachricht an das Observable und der Live-Feed
	      abonniert dieses Observable, wodurch es die neue Nachrichtenliste automatisch erhält, sobald diese sich ändert. Dadurch wird eine effektive
	      Brücke zwischen den beiden Komponenten geschlagen.

	\item \textbf{Marker Service:}
	      Der Marker Service dient der Verwaltung der Pilzfundort-Markierungen auf der Karte. Er erhält die angezeigte Karteninstanz und kann dadurch auf
	      dieser temporäre Markierungen setzen. Außerdem enthält er Methoden zum Persistieren eines Markers, wenn der User einen Pilz einträgt. Somit ist
	      auch dieser Service essentiell, um das Zusammenspiel zwischen der Karten- und Eingabemaskenkomponente zu ermöglichen.

	\item \textbf{Shroom Service:}
	      ...

\end{itemize}

\subsubsection{Backend Kommunikation}

\subsubsection{Responsive Design}

Die Anpassung der HTML Templates an kleinere Bildschirmgrößen wurde ausschließlich in den CSS-Dateien über die `@media'-Query geregelt. Dort können für
selbst festgelegte Bildschirmbreiten und -höhen andere CSS-Styles festgelegt werden, als normalerweise gelten. Somit wurden durch wiederholtes Anpassen
und Ausprobieren alle Teile der Applikation so gestaltet, dass sie bei jeder Bildschirmgröße sichtbar und sinnvoll platziert sind.

\end{document}
