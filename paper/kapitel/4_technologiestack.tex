\documentclass[../main.tex]{subfiles}
%~1800 Worte
\begin{document}
\subsection{Uebergreifende Technologien} %Lauritz
\subsubsection{VSCode}
Visual Studio Code, kurz VSCode, ist eine integrierte Entwicklungsumgebung (IDE), die von Microsoft entwickelt wurde. 
Eine IDE ist eine Software, die umfassende Funktionen zur Softwareentwicklung unter einer Oberfläche bündelt, einschließlich 
eines Quelltext-Editors, Debugging-Tools und oft auch Möglichkeiten zur Versionskontrolle. Die Entscheidung für VSCode als 
primäre Entwicklungsumgebung für "ShroomScout" fiel aufgrund mehrerer Faktoren. Zum einen bringen beide Entwickler bereits 
umfassende Erfahrung mit VSCode aus früheren Projekten mit, was eine effiziente Arbeit ermöglicht. Zum anderen bietet VSCode 
eine einfache Erweiterbarkeit durch zahlreiche Extensions, die speziell für die Arbeit mit Angular, Git und anderen relevanten 
Technologien konzipiert sind. Diese Erweiterungen erleichtern die Entwicklung erheblich, indem sie zusätzliche Funktionen wie 
Syntaxhervorhebung, Codevervollständigung und direkte Integration mit Git bereitstellen.

\subsubsection{Git & GitHub}
Git ist ein weit verbreitetes System zur Versionskontrolle, das es Entwicklern ermöglicht, Änderungen am Quellcode nachzuverfolgen 
und die Zusammenarbeit in Entwicklungsteams zu erleichtern. Git unterstützt die verteilte Arbeit, indem es jedem Entwickler ermöglicht, 
eine vollständige Kopie des Repositorys lokal zu führen, inklusive der gesamten Historie der Codeänderungen. Für "ShroomScout" wurde 
Git genutzt, um den Entwicklungsprozess zu verwalten und Änderungen effizient zu koordinieren. Das Projekt wurde auf GitHub gehostet, 
einer Plattform, die kostenlose Hosting-Services für Git-Repositorys bietet. GitHub erleichtert die Zusammenarbeit, indem es Werkzeuge 
für die Problemverfolgung, Code-Reviews und die Verwaltung von Pull Requests bereitstellt. Die Nutzung von GitHub als zentrale Plattform 
trug zur reibungslosen Koordination zwischen den Entwicklern bei und ermöglichte eine transparente und nachvollziehbare Projektentwicklung.

\subsubsection{Open Street Map}
Open Street Map (OSM) ist ein kollaboratives Projekt zur Erstellung einer frei bearbeitbaren Weltkarte. Im Gegensatz zu kommerziellen 
Kartenanbietern wie Google Maps bietet OSM den Vorteil, dass die Daten unter einer offenen Lizenz stehen, was die freie Nutzung und 
Anpassung der Karteninformationen ermöglicht. Für "ShroomScout" wurde OSM als Grundlage für die interaktive Karte gewählt, auf der Nutzer 
Pilzfundorte eintragen können. Ein entscheidender Vorteil von OSM ist die Möglichkeit, eigene Layer zu erstellen und zu verwalten, was es 
ermöglicht, Pilzfundorte präzise zu dokumentieren und visuell hervorzuheben. Diese Flexibilität und die offene Datenlizenz machen OSM zu 
einer idealen Wahl für Projekte, die auf detaillierte geographische Informationen angewiesen sind und gleichzeitig die Kontrolle über die 
Darstellung und Verwaltung der Kartendaten behalten möchten.

\subsection{Datenbankloesung} %Tim
\subsubsection{SQLite}
\subsubsection{Tabellenstruktur}

\subsection{Backend Technologien} %Tim
\subsubsection{FastAPI mit Python}
\subsubsection{Uvicorn ASGI Server}

\subsection{Frontend Technologien} %Lauritz
\subsubsection{Angular}
Angular ist ein umfassendes Frontend-Framework, entwickelt und unterstützt von Google. Es ermöglicht Entwicklern die Erstellung dynamischer 
Single-Page-Applications (SPAs). Angular basiert auf TypeScript, einer von Microsoft entwickelten Programmiersprache, die eine Erweiterung 
von JavaScript darstellt. TypeScript bietet starke Typisierung und objektorientierte Programmierkonzepte, was die Entwicklung großer und 
komplexer Anwendungen erleichtert und die Codequalität verbessert. In Kombination mit HTML und CSS ermöglicht Angular die Entwicklung von 
Anwendungen, die sowohl funktional als auch ästhetisch ansprechend sind.



Die Architektur von Angular ist komponentenbasiert, was bedeutet, dass Anwendungen aus wiederverwendbaren Komponenten aufgebaut sind, die 
jeweils einen Teil der Benutzeroberfläche darstellen. Diese Komponenten können unabhängig voneinander entwickelt, getestet und in der 
Anwendung wiederverwendet werden, was die Entwicklungseffizienz und -geschwindigkeit erhöht.

Eines der Hauptmerkmale von Angular ist die Fähigkeit zur Entwicklung von Single-Page-Applications (SPAs). SPAs laden bei der Initialisierung 
die gesamte Anwendung einmalig und aktualisieren dann Inhalte dynamisch ohne vollständige Seite-Neuladungen. Dies führt zu einer erheblich 
verbesserten Benutzererfahrung, da Nutzer nahtlose Übergänge zwischen verschiedenen Teilen der Anwendung erleben, was die Reaktionsgeschwindigkeit 
und das Gefühl einer Desktop-Anwendung im Webbrowser nachahmt.

\subsubsection{Komponentenhierarchie}
\subsubsection{Messageflow \& RxJS}
\end{document}

